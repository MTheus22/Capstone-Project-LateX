
\documentclass{configs/idp-model}
\usepackage{graphicx}
\usepackage{wallpaper}
\usepackage{}


%PREENCHER TODOS OS CAMPOS ABAIXO
\autor{Linus Torvalds}


% Utilizar um dos dois cursos abaixo
\cienciadacomputacao
%\engenhariadesoftware


\titulo{Um título muito interessante para um trabalho}
\ano{2025}

% Usar orientador ou orientadora
\orientador{Dr. Alan Turing}
%\orientadora{Dra. Ada Lovelace}

% Para membros da banca, deve-se por o nome no primeiro parâmentro e se é examinador interno ou externo no segundo parâmetro
\membrobancai{Dr. Dom Quixote de la Mancha}{Examinador interno}
\membrobancaii{Dra. Anna Karienina}{Examinadora Externa}

\palavraschave{LaTeX, metodologia científica, trabalho de conclusão de curso, artigo, tecnologia}
\keywords{{LaTeX, metodologia científica, trabalho de conclusão de curso, paper, technology}}

\dataaprovacao{08/05/2025}



\begin{document}

% Não alterar
\pagenumbering{roman}
\capaprincipal
\folharosto
\folhaaprovacao

% Opcional
\dedicatoria
\agradecimentos


% Não alterar 
% Resumo e abstract - alterar os arquivos na pasta "partes" NÃO ALTERAR O NOME DELES
\abstract
\resumo

%List of Figures e Tables são optativos. Caso não esteja utilizando no seu projeto, comentar as linhas abaixo
\listoffigures
\listoftables

\tableofcontents
\pagenumbering{arabic}


% Para as seções, o comando \secao deve ter a forma \secao{título}{arquivo .tex}

% Os comandos \subsecao e \subsubsecao seguem a forma \comando{Título}, sem a necessidade de passar um arquivo .tex

\secao{Introducao}{partes/introducao}
\secao{Fundamentação Teórica}{partes/fundamentacao}
\secao{Revisão Sistemática}{partes/rsl}
\secao{Metodologia}{partes/desenvolvimento}
\secao{Conclusão}{partes/conclusao}

% Não alterar
\referencias

% Opcionais
\apendice{Título do apêndice A}{partes/apendicei}

\anexo{Título do Anexo A}{partes/anexoi}


% Não alterar
\capafinal

 
\end{document}
