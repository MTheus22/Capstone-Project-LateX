
Os anexos são documentos, materiais ou informações que \textbf{não foram elaborados pelo autor do trabalho}, mas que servem para complementar, ilustrar ou fundamentar o conteúdo apresentado.

Exemplos comuns de anexos incluem:

\begin{itemize}
    \item Legislações, normas técnicas ou regulamentos;
    \item Manuais, folhetos, publicações oficiais;
    \item Textos, tabelas ou documentos externos que deram suporte à pesquisa;
    \item Materiais gráficos ou audiovisuais cedidos por terceiros.
\end{itemize}

Assim como os apêndices, os anexos são organizados em sequência e identificados por letras maiúsculas (Anexo A, Anexo B, etc.), com títulos claros.

Eles são inseridos após os apêndices (quando houver) ou diretamente após as referências bibliográficas, e devem ser referenciados no corpo do trabalho para que o leitor saiba quando consultá-los.
