A revisão bibliográfica tem como objetivo apresentar o embasamento teórico que sustenta o desenvolvimento do trabalho. Por meio dela, são explorados os conceitos, teorias, modelos e estudos já existentes relacionados ao tema. Esta etapa permite contextualizar o problema de pesquisa, identificar lacunas no conhecimento e justificar a realização do estudo.

A revisão deve ser elaborada a partir de fontes confiáveis, como livros, artigos científicos, dissertações, teses e documentos técnicos. É fundamental que os materiais selecionados sejam atuais, relevantes e diretamente relacionados ao objeto de estudo.

A seguir, esta seção está dividida em duas partes: \textit{Fundamentação Teórica} e \textit{Trabalhos Correlatos}.

\subsecao{Theoretical Framework}

Na fundamentação teórica, são apresentados os conceitos, definições e teorias que servem de base para o desenvolvimento do trabalho. Este conteúdo deve ser organizado de maneira lógica, agrupando os temas de acordo com a necessidade do estudo.

É nessa subseção que se aprofunda o entendimento sobre o tema principal, explorando os modelos, princípios, leis, metodologias ou qualquer outro arcabouço teórico relevante. Todo o conteúdo deve estar devidamente referenciado, demonstrando que o trabalho está alinhado ao conhecimento científico já produzido.

\subsecao{Related Works}

Nesta subseção, são apresentados estudos, projetos, pesquisas ou soluções já desenvolvidas que possuem relação direta ou indireta com o tema proposto.

O objetivo é identificar como outros autores trataram problemas semelhantes, quais métodos utilizaram, quais resultados alcançaram e quais limitações foram encontradas. A análise dos trabalhos correlatos permite destacar as contribuições existentes na área, bem como evidenciar lacunas que seu trabalho busca preencher.

A apresentação dos trabalhos correlatos deve ser feita de forma crítica, não se limitando à simples descrição dos estudos, mas também comparando-os entre si e relacionando-os com o problema de pesquisa.