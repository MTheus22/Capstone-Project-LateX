
Esta seção tem como objetivo apresentar, analisar e interpretar os resultados obtidos na pesquisa. Os dados devem ser organizados de forma clara, utilizando tabelas, gráficos, quadros ou figuras, quando necessário, para facilitar a compreensão.

A análise dos resultados deve ser feita de forma crítica, relacionando-os com os objetivos propostos e com o referencial teórico previamente abordado. É fundamental interpretar o significado dos dados, verificando se eles corroboram ou refutam as hipóteses levantadas ou os estudos anteriores consultados.

Nesta etapa, não basta apenas descrever os resultados; é necessário discutir suas implicações, possíveis limitações, e o que eles revelam sobre o problema de pesquisa. A discussão pode ser apresentada juntamente com os resultados ou em um capítulo separado, conforme as normas da instituição ou a natureza do trabalho.

Portanto, esta seção deve responder às seguintes perguntas:

\begin{itemize}
    \item Quais foram os principais resultados obtidos?
    \item Como esses resultados se relacionam com os objetivos do trabalho?
    \item Eles confirmam ou refutam o que foi encontrado na literatura?
    \item Quais são as possíveis explicações para os achados?
    \item Existem limitações nos dados ou na metodologia que influenciam os resultados?
\end{itemize}

A apresentação dos resultados e sua discussão devem ser feitas de maneira clara, objetiva e fundamentada, garantindo coerência com todo o desenvolvimento do trabalho.