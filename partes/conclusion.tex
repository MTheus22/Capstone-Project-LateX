
A conclusão é a seção final do trabalho, onde o autor retoma os objetivos propostos no início e apresenta as considerações finais com base nos resultados alcançados. Seu papel é responder se o problema de pesquisa foi solucionado e se os objetivos foram atingidos.

Neste capítulo, não se introduzem novos dados, informações ou discussões. O foco está em:

\begin{itemize}
    \item Retomar brevemente os objetivos do trabalho;
    \item Apontar se os objetivos foram alcançados;
    \item Destacar os principais resultados obtidos;
    \item Apresentar reflexões sobre as limitações do estudo, quando houver;
    \item Sugerir possíveis desdobramentos, melhorias ou pesquisas futuras sobre o tema.
\end{itemize}

A conclusão deve ser clara, objetiva e coerente com tudo o que foi desenvolvido ao longo do trabalho, fechando o raciocínio de forma lógica e consistente.

