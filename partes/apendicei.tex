
Os apêndices são materiais suplementares \textbf{elaborados pelo próprio autor}, que contribuem para a compreensão, fundamentação ou detalhamento do trabalho, mas que não são essenciais para o desenvolvimento principal do texto.

Eles podem incluir, por exemplo:

\begin{itemize}
    \item Questionários e roteiros de entrevistas utilizados na pesquisa;
    \item Documentos, códigos-fonte, tabelas extensas ou informações complementares;
    \item Detalhes técnicos ou cálculos que, se inseridos no corpo principal, tornariam a leitura cansativa;
    \item Materiais que apoiam a metodologia ou os resultados, mas que não são obrigatórios para a compreensão geral.
\end{itemize}

Cada apêndice deve ser identificado por uma letra maiúscula (Apêndice A, Apêndice B, etc.) e ter um título claro e objetivo. Eles são inseridos após as referências bibliográficas e antes do índice remissivo, se houver.

A inclusão dos apêndices deve ser feita de maneira organizada e referenciada no texto principal, para que o leitor saiba quando e por que consultar esses materiais complementares.