A metodologia descreve, de forma clara e detalhada, os procedimentos adotados para o desenvolvimento da pesquisa. É nesta seção que se explica como o estudo foi conduzido, permitindo que outros pesquisadores possam replicar ou compreender os passos realizados.

Devem ser apresentados os seguintes elementos, quando aplicável:

\begin{itemize}
    \item \textbf{Tipo de pesquisa}: definir se é qualitativa, quantitativa ou quali-quantitativa, além de classificar como exploratória, descritiva, explicativa, estudo de caso, pesquisa de campo, bibliográfica, documental, entre outras.
    
    \item \textbf{Procedimentos metodológicos}: descrever como foi realizado o levantamento de dados, incluindo instrumentos (questionários, entrevistas, observações, análises documentais, softwares, entre outros) e etapas do processo.
    
    \item \textbf{População e amostra}: identificar o universo da pesquisa, bem como os critérios de seleção da amostra, se houver.
    
    \item \textbf{Análise dos dados}: indicar quais métodos, técnicas estatísticas ou ferramentas foram utilizados para o tratamento e análise dos dados coletados.
    
    \item \textbf{Materiais e métodos}: em pesquisas aplicadas ou experimentais, especificar os materiais, equipamentos, ferramentas e procedimentos técnicos utilizados.
\end{itemize}

A metodologia deve ser apresentada de forma objetiva, precisa e suficientemente detalhada, garantindo a confiabilidade e a possibilidade de reprodução do estudo.
