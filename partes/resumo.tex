O Resumo é uma apresentação concisa do conteúdo do trabalho. Deve sintetizar, de forma clara e objetiva, os principais elementos do texto: tema, objetivo, metodologia, resultados e conclusões. O objetivo do resumo é permitir que o leitor compreenda rapidamente o que foi desenvolvido no trabalho, sem precisar lê-lo na íntegra. De acordo com as normas da ABNT (NBR 6028), o resumo deve ser elaborado em um único parágrafo, sem citações, com a utilização de linguagem impessoal e verbo na voz ativa. Geralmente, possui entre 150 e 500 palavras, dependendo do regulamento da instituição. Além do texto, o resumo deve apresentar de três a cinco palavras-chave, separadas por ponto e finalizadas com ponto final, que representem os principais assuntos abordados no trabalho.